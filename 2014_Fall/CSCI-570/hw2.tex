%%%%%%%%%%%%%%%%%%%%%%%%%%%%%%%%%%%%%%%%%
% Structured General Purpose Assignment
% LaTeX Template
%
% This template has been downloaded from:
% http://www.latextemplates.com
%
% Original author:
% Ted Pavlic (http://www.tedpavlic.com)
%
% Note:
% The \lipsum[#] commands throughout this template generate dummy text
% to fill the template out. These commands should all be removed when 
% writing assignment content.
%
%%%%%%%%%%%%%%%%%%%%%%%%%%%%%%%%%%%%%%%%%

%----------------------------------------------------------------------------------------
%	PACKAGES AND OTHER DOCUMENT CONFIGURATIONS
%----------------------------------------------------------------------------------------

\documentclass{article}

\usepackage{fancyhdr} % Required for custom headers
\usepackage{lastpage} % Required to determine the last page for the footer
\usepackage{extramarks} % Required for headers and footers
\usepackage{graphicx} % Required to insert images
\usepackage{lipsum} % Used for inserting dummy 'Lorem ipsum' text into the template

\usepackage{amsmath}
%\usepackage[]{algorithm2e}
%\usepackage{algpseudocode}
\usepackage{verbatim}
%\usepackage{algorithm}
%\usepackage[noend]{algpseudocode}
\usepackage[]{algorithm2e}


% Margins
\topmargin=-0.45in
\evensidemargin=0in
\oddsidemargin=0in
\textwidth=6.5in
\textheight=9.0in
\headsep=0.25in 

\linespread{1.1} % Line spacing

% Set up the header and footer
\pagestyle{fancy}
\lhead{\hmwkAuthorName} % Top left header
\chead{\hmwkClass\ : \hmwkTitle} % Top center header
\rhead{\firstxmark} % Top right header
\lfoot{\lastxmark} % Bottom left footer
\cfoot{} % Bottom center footer
\rfoot{Page\ \thepage\ of\ \pageref{LastPage}} % Bottom right footer
\renewcommand\headrulewidth{0.4pt} % Size of the header rule
\renewcommand\footrulewidth{0.4pt} % Size of the footer rule

\setlength\parindent{0pt} % Removes all indentation from paragraphs

%----------------------------------------------------------------------------------------
%	DOCUMENT STRUCTURE COMMANDS
%	Skip this unless you know what you're doing
%----------------------------------------------------------------------------------------

% Header and footer for when a page split occurs within a problem environment
\newcommand{\enterProblemHeader}[1]{
\nobreak\extramarks{#1}{#1 continued on next page\ldots}\nobreak
\nobreak\extramarks{#1 (continued)}{#1 continued on next page\ldots}\nobreak
}

% Header and footer for when a page split occurs between problem environments
\newcommand{\exitProblemHeader}[1]{
\nobreak\extramarks{#1 (continued)}{#1 continued on next page\ldots}\nobreak
\nobreak\extramarks{#1}{}\nobreak
}

\setcounter{secnumdepth}{0} % Removes default section numbers
\newcounter{homeworkProblemCounter} % Creates a counter to keep track of the number of problems

\newcommand{\homeworkProblemName}{}
\newenvironment{homeworkProblem}[1][Problem \arabic{homeworkProblemCounter}]{ % Makes a new environment called homeworkProblem which takes 1 argument (custom name) but the default is "Problem #"
\stepcounter{homeworkProblemCounter} % Increase counter for number of problems
\renewcommand{\homeworkProblemName}{#1} % Assign \homeworkProblemName the name of the problem
\section{\homeworkProblemName} % Make a section in the document with the custom problem count
\enterProblemHeader{\homeworkProblemName} % Header and footer within the environment
}{
\exitProblemHeader{\homeworkProblemName} % Header and footer after the environment
}

\newcommand{\problemAnswer}[1]{ % Defines the problem answer command with the content as the only argument
\noindent\framebox[\columnwidth][c]{\begin{minipage}{0.98\columnwidth}#1\end{minipage}} % Makes the box around the problem answer and puts the content inside
}

\newcommand{\homeworkSectionName}{}
\newenvironment{homeworkSection}[1]{ % New environment for sections within homework problems, takes 1 argument - the name of the section
\renewcommand{\homeworkSectionName}{#1} % Assign \homeworkSectionName to the name of the section from the environment argument
\subsection{\homeworkSectionName} % Make a subsection with the custom name of the subsection
\enterProblemHeader{\homeworkProblemName\ [\homeworkSectionName]} % Header and footer within the environment
}{
\enterProblemHeader{\homeworkProblemName} % Header and footer after the environment
}
   
%----------------------------------------------------------------------------------------
%	NAME AND CLASS SECTION
%----------------------------------------------------------------------------------------

\newcommand{\hmwkTitle}{Homework\ \# 2 } % Assignment title
\newcommand{\hmwkDueDate}{Friday,\ September\ 12 ,\ 2014} % Due date
\newcommand{\hmwkClass}{CSCI-570} % Course/class
\newcommand{\hmwkAuthorName}{Saket Choudhary} % Your name
\newcommand{\hmwkAuthorID}{2170058637} % Teacher/lecturer
\newcommand{\hmwkAuthorEmail}{skchoudh@usc.edu} % Teacher/lecturer
%----------------------------------------------------------------------------------------
%	TITLE PAGE
%----------------------------------------------------------------------------------------

\title{
\vspace{2in}
\textmd{\textbf{\hmwkClass:\ \hmwkTitle}}\\
\normalsize\vspace{0.1in}\small{Due\ on\ \hmwkDueDate}\\
%\vspace{0.1in}\large{\textit{\hmwkClassTime}}
\vspace{3in}
}

\author{\textbf{\hmwkAuthorName} \\
	\textbf{\hmwkAuthorEmail}\\
	\textbf{\hmwkAuthorID}
	}
\date{} % Insert date here if you want it to appear below your name

%----------------------------------------------------------------------------------------

\begin{document}

\maketitle

%----------------------------------------------------------------------------------------
%	TABLE OF CONTENTS
%----------------------------------------------------------------------------------------

%\setcounter{tocdepth}{1} % Uncomment this line if you don't want subsections listed in the ToC

\newpage
\tableofcontents
\newpage


%----------------------------------------------------------------------------------------
%	PROBLEM 2
%----------------------------------------------------------------------------------------

\begin{homeworkProblem}[HW2] % Custom section title

\begin{comment}

\begin{homeworkSection}{(2: Ch\#2 Ex\#3)} 
\problemAnswer{
	\textbf{Part (a)} $n^2$
	
	Doubling the input size make it slower by $\frac{(2n)^2}{n^2} = 4$
	
	Consider increasing input size by 1: $ \frac{(n+1)^2}{n^2} = \frac{n^2+2n+1}{n^2} = 1 + \frac{1}{n} + \frac{1}{n^2}$
	
	For $\lim_{n\to\infty}$, $1 + \frac{1}{n} + \frac{1}{n^2} = 1$, Thus the algorithm with $n+1$ input is as slow as with input size $n$ for $n\to \infty$.
	
	
	\textbf{Part (b): } $n^3$
	
	Doubling the input size: $\frac{(2n)^3}{n^3} = 8$, thus it is 8 times slower.
	
	Increasing the input size by 1: $\frac{(n+1)^3}{n^3} = \frac{n^3+3n^2+3n+1}{n^3} = 1 + \frac{3}{n} + \frac{3}{n^2} + \frac{1}{n^3} $.
	
	For $\lim_{n\to\infty}$, $1 + \frac{3}{n} + \frac{3}{n^2} + \frac{1}{n^3} = 1$.
	
	\textbf{Part (c):} This is similar to Part(a), since the factor of 100 is common. The solution is exactly similar to Part(a)
	
	\textbf{Part (d): } $n log n$
	
	Doubling the input size: $\frac{2n log 2n}{n log n} = \frac{2 log 2n }{log n}$. For  $\lim_{n\to\infty}$, this would blow up.
	
	Increasing the input size by 1: $\frac{n+1 log (n+1)}{n log n} $.
	
	For $\lim_{n\to\infty}$, $\frac{n+1 log (n+1)}{n log n}  = 1$.
	
	Hence input with $n+1$ is as slow as $n$ $\lim_{n\to\infty}$
	
	\textbf{Part (e):} $2^n$
	
	Doubling the input size: $\frac{2^{2n}}{2^n} = 2^n$, which blows up as $\lim_{n\to\infty}$.
	
	Increasing the input size by 1 : $\frac{2^(n+1)}{2^n} = 2$. 
	
	Thus increasing the input size by 1 causes it to be 2 times slower.
	
	
	 
}
\end{homeworkSection}

\begin{homeworkSection}{(3: Ch\#2 Ex\#4)} %
	\problemAnswer{
			  \textbf{Given:} Operating speed = $10^10$ operations per second. 
			  
			  \textbf{To Find: }Maximum possible n, for 3600s operation
			  
			  \textbf{Part (a): } $n^2$
			  
			  $n^2 = 36 * 10^12$ $\implies$ $n = 6 * 10^6$
			  
			  \textbf{Part (b):} $n^3$
			  
			  $n^3 = 36*10^12$ $\implies$ $ n= (36)^0.333 * 10^4 $
			  
			  \textbf{Part (c):} $100n^2$
			  
			  $100n^2 = 36 * 10^12 \implies n = 6*10^5$
			  
			  \textbf{Part (d):} $n log n$
			  
			  $log (n^n) = 36*10^12$ %TODO
			  
			  \textbf{Part (e):} $2^{2^n}$
			  
			  $2^{2^n} = 36 * 10^12 \implies n = log_2(log_2(36*10^12))$
			  
			   
		}
	
\end{homeworkSection}	
\end{comment}
\begin{homeworkSection}{(2: Ch\#2 Ex\#3)} %
	\problemAnswer{
		$f1=n^2.5, f2=\sqrt{2n}, f3=n+10, f4=10^n, f5=100^n, f6=n^2logn$
		
		Consider the square of $f_i$: \\
		
		$f1'= n^5; f2 = 2n; f3 = (n+10)^2; f4=10^{2n}; f5=100^{2n};
		f6 = n^4 (log n)^2$
		
		Since exponentials always grow faster than polynomials, in the order of running time complexity higher to lower:
		
		$100^2n > 10^2n > n^5 > n^4 (log n)^2 > (n+2)^2 > n2n$

		Thus:
		
		$ f5 > f4 > f1 > f6 > f3 > f2 $
		}
\end{homeworkSection}

\begin{homeworkSection}{(3: Ch\#2 Ex\#4)}
	\problemAnswer{
		
		$g1= 2^{sqrt{log n}}; g2 = 2^n; g3 = n(log n)^3; g4 = n^{\frac{4}{3}}; g5 = n^{log n}; g6 = 2^{2^n}; g7 = 2^{n^2}$
		
		Since exponentials grow faster than polynomials:
		$g1, g2, g6, g7$ are definitely asymptotically larger than the rest.
		
		$2^{2^n} > 2^{n^2} > 2^{n} > 2^{\sqrt{log n}}$
		
		Considering $g3, g4, g5$: \\
		
		$n^{log n} > n^{\frac{4}{3}} > n(log n)^3$
		
		Thus: \\
		
		$g6 > g7 > g2 > g1 > g5 > g4 > g3 $
		
		 
		
		}
\end{homeworkSection}

\begin{homeworkSection}{(4: Ch\#3, Ex\#5 )} % Section within problem
%\lipsum[4]\vspace{10pt} % Question

\problemAnswer{ 
	\textbf{Given:} $f(n) = O(g(n)) \implies f(n) \leq cg(n) for\  c>0, \forall n \neq n_0$
	
	\textbf{Part (a): }  $log_2(f(n))\ is\ O(log_2(g(n)))$
	
	$f(n) \leq cg(n) for\ c>0, \forall n \neq n_0$
	
	As $g(n)$ is positiev defnite, a lesser strict bound on f(n) is given by: \\
	
	%$f(n) \leq g(n)^d for\ d > 1, \forall n > n_0$. Taking %logarithm(since it is a monotonic function) on both sides we get: \\
	
	Taking logarithm on both sides: 
		$ log_2 (f(n)) \leq log_2(g(n)) + log_2(c)$  for  $c>0, \forall n > n_0$
		
		if $|g(n)| > 2$ the RHS would be positive definite, and
		$log_2(f(n))= O(log_2(g(n)))$  would be true. But this is not true in general($|g(n)| <2 $)
		
	
	$ log_2 (f(n)) \leq log_2(g(n)^d)$ $\implies$ $ log_2(f(n)) \leq d log_2(g(n))$, for some $d>1, \forall n > n_0$
	
	Which clearly proves Part (a), but is a special case and 
	hence \textbf{(a) is FALSE in general.}
	
	
	\textbf{Part (b)} $2^{f(n)}\ is\ O(2^{g(n)})$
	
%	Since, 	$f(n) \leq cg(n) for\ c>0, \forall n \neq n_0$,
%	taking exponentials on both sides: \\
%		$ 2^{f(n)} \leq 2^{cg(n)} for\ c>0, \forall n \neq n_0$ 
%		$\implies$ 		$ 2^{f(n)} \leq {2^{g(n)}c} for\ c>0, \forall n \neq n_0$ 
		
		% TODO
		
		Consider $f(n) = 2n^2$, and $g(n) = n^2$, then 
		$2^{f(n)} = 2^{2n^2}$ while $2^{g(n)} = 2^{n^2}$, which clearly does not satisfy the given relation.
		
	\textbf{	Hence FALSE.}
		
	\textbf{Part (c)} 
	
		Since, 	$f(n) \leq cg(n) for\ c>0, \forall n \neq n_0$,
		squaring both sides: \\
		$f(n)^2 \leq c^2g(n)^2 for\ c>0, \forall n \neq n_0$
		 $\implies$ $f(n)^2 \leq c'g(n)^2 for\ c'>0, \forall n \neq n_0$
		 
		\textbf{Thus Part(C) is TRUE.}

}

\end{homeworkSection}

%--------------------------------------------

\begin{homeworkSection}{(5: Ch\#2, Ex\#6 )} % Section within problem
\problemAnswer{
\textbf{Part (a, b): } There is one loop involved besides the for i,j loop that calculates the sum of A[i] through A[j], which is equal to $\sum_i \sum_j (j-i)$ = $\sum_i^{n} \sum_{j=i+1}{n} (j-i) = \sum_i^{n} n(n+1)/2 = O(n^3)$

\textbf{Part (c):} To avoid summing A[i] through A[j], we rely on $B[i,j] = B[i, j-1] + A[j]$ initialising B[1,1] as A[1].


}

\end{homeworkSection}
\begin{homeworkSection}{(6: Ch\#3, Ex\#4 )}
	\problemAnswer{
		We assume the given statement to be true.
		Consider n nodes with two children each and m leaves. The Binary tree is not necessarily balanced. $n-m \leq 1$
		
		Adding one more node to this tree will lead to $n+1$ nodes. Now there are two scenarios:
		
		\textbf{Case1 :} The binary tree was balanced and adding one node lead to an imbalanced tree.
		
		This will cause the number of leaves to remain same,'m' as the new node is entirely at a new level(depth), thus accounting for one leave and the nodes at the penultimate level are all leaves except the one whose child is this new node. So we have $m'=1 + m-1=m$ leaves and $n'=n$ nodes with two children. Since $n-m \leq 1 \implies n'-m' \leq 1$. Adding a new leaf to a balanced binary tree does not lead to either more leaves or more nodes with two children.(the added leaf's parent has a single child!)
		
		\textbf{Case2} The binary tree was imbalanced so the new node is added to the same level as the leaves were. Thus the number of leaves go up by 1. The node to which this is added may transform into a parent with two or one child and so $n'= n\ OR\ n+1$ depending on where the new leaf is added. and $m'=m+1$ in either cases. Since $n-m \leq 1 \implies (n-(m+1) \leq 0$; $(n+1-(m+1) \leq 1)$, Thus $n'-m' \leq 1$
		
		Hence Proved.
		  }
\end{homeworkSection}

\begin{homeworkSection}{(7: Ch\#3, Ex\#6 )}
	\problemAnswer{
		Let $(x, y) \in G$ be an edge such that $(x,y) \not\in T$, G rooted at $u$. Since $(x, y)$ are connected they must occur in some layer of $T$. 
		Since $T$ is a DFS tree, ${x,y}$ must be ancestor of each other. Say $x$ is an ancesotr of $y$. At the same time $T$ is a BFS tree, and hence the difference of distance of $(s,x)$
and $(s,y)$ can differ at maximum by $1$ and they are ancestors, so they should be connected in $T$ via an edge too. 	}
\end{homeworkSection}

\begin{homeworkSection}{(8)}
	\problemAnswer{
		Let the depth of DFS tree $T$ be less than that of BFS tree $U$. Now consider the leaf of $T$ can then be reached from $v$ in a shorter distance than in $U$ since the depth of $T$ is smaller, which is a contradiction to $BFS$ tree property of providing the shortest distances.
		}
\end{homeworkSection}
%--------------------------------------------

\end{homeworkProblem}



\end{document}
