\documentclass{article}
\usepackage[utf8]{inputenc}
\usepackage{fullpage}
\usepackage{amsmath}
\usepackage{amsmath}

%\usepackage{amsymbols}

\begin{document}
\section{Review}
Change of Bases: 
$$
\mathcal{B} \{b_1, b_2, \dots b_n \} \longrightarrow B \{b_1|b_2\dots|b_n\}
$$
$$
\mathcal{C} \{c_1, c_2, \dots c_n \} \longrightarrow C \{c_1|c_2\dots|c_n\}
$$


Diagnolization:
$$
A \in C^{n \times n}
$$

$A$ is diagnolizable if $\exists$ $X \in C^{n\times n}$ duch that $det (X) \neq 0$ 

$v \in C^{n \times n}$ 

$ [v] = B[v]_B$ 

$[v]_B = B^{-1}[v]$ 

$[v]_C = C^{-1}B[v]_B$


$A = X\Lambda X^{-1}$ , $\Lambda = diag(\lambda_1, \lambda_2, \dots \lambda_n)$
$\Lambda = X^{-1}AX$


Diagnolization is like change of basis

$$
A[v] = X\Lambda X^{-1}[v]/[v]_X = [\Lambda v]_X/[\Lambda v]
$$

\textbf{Claim: }If spectrum of $A$, $card(\sigma(A)) = n i.e.(\lambda_i \neq \lambda_j, i \neq j)$ then $A$ is diagonizable(non-defective)

\textbf{Proof:} Do induction on \# of eigen vector

$\lambda_1 \neq \lambda_2$
$c_1\vec{X}_1 + c_2\vec{X}_2 = 0$; $c_1,c_2 \neq 0$ 

$c_1A\vec{X}_1 + c_2A\vec{X}_2 =0$

$c_1\lambda_1\vec{X}_1 + c_2\lambda_2\vec{X}_2 =0$

$-c_1\lambda_2\vec{X}_1 + c_2\lambda_2\vec{X}_2 =0$

 $c_1(\lambda_1-\lambda_2)\vec{X}_1=0$ $->$ Contradiction
 
 Assume truth for $k=1$
 
 $\sum_{i=1}^k c_i \vec{X}_i = 0$ $c_i is not zero$
 
 In particular, at least one of $c_i$ is not zero
 
 \begin{align*}
\sum_{i}c_i\lambda_i X_i &= 0\\
0 &= \lambda_k 0 - 0\\
&= \lambda_k \sum c_iX_o - \sum_i c_i \lambda_i X_i\\
&= \sum_i c_i(\lambda_k - \lambda_i)X_i\\ contradiction
\end{align*}

$X_i$ are independent, hence follows.

\section{Interpretation}
Assume $A \in C^{n \times n} \text{ and } \vec{b} \in C^n$ ; $det(A) \neq 0$

$AX = \vec{b}$

Best case:
\begin{itemize}
\item $A$ is diagnolizable
\item $A$ is triangular(upper/lower)
\end{itemize}

Properties:
\begin{itemize}
\item Production of 2 upper triangle is upper triangular
\item Inverse of non singular upper triangular is upper triangular
\end{itemize}

Argument of (2):
\begin{align*}
SX &= I\\
[SX_1, SX_2, \dots, SX_n] &= [e_1,e_2, \dots e_n] (Std basis)\\
S\vec{X}_i &= \vec{e}_i
\end{align*}

Elementary row operations

\begin{itemize}
\item R1 Multiply on RHS by a constant
\item R2 Exchanging two rows
\item R3 add non zero multiple of one to another
\end{itemize}

(R2)Permutation matrix: exactly one 1 in row or column -> Not lower or upper traingular. But holds for R1,R3



LU decomposition $[L_1, L_2,L3]A = U$ $\hat{L}A = U$
$A = LU$
$L=\hat{L}^{-1}$

$AX =b -> Ly= b -> UX=y$

Two factorization:
1. $A = X\Lambda X^{-1}$ when $X$ is non-defective
2. $A = LU$, when $A$ is square and not of permutation type.


\section{Norms}

Absolute value:
\begin{align*}
|a| &\geq 0 \\
|a| &=0 -> a=0,\\
|ab| & = |a||b|\\
|x+y| \leq |x|+|y|
\end{align*}

Norm:
\begin{itemize}
\item Norm is  mapping $||. || : V -> R$ $V over C$
\item ||0||: V ->R
\item $||\vec{v}|| \geq 0, ||v|| = 0$ iff $v=0$
\item $||c\vec{v}|| = |c|||\vec{v}||$
\item $||\vec{v}+\vec{w}|| \leq ||\vec{v}|| + ||\vec{w}||$
\end{itemize}

$p$ norm:
$p \leq \infty$

$||\vec{v}||_p = \sum (|v_i|^p)^{1/p}$

$||v||_{\infty} = max|V_i|$ $1 \leq i \leq n$ 


\textbf{Clasim:} $||v||_p$ is a norm
\end{document}
