\documentclass{article}
\usepackage{times,amsmath,amsthm,amsfonts,eucal,graphicx, amssymb}
 \setlength{\oddsidemargin}{0.25 in}
\setlength{\evensidemargin}{-0.25 in}
\setlength{\topmargin}{-0.6 in}
\setlength{\textwidth}{6.5 in}
\setlength{\textheight}{8.5 in}
\setlength{\headsep}{0.75 in}
\setlength{\parindent}{0 in}
\setlength{\parskip}{0.1 in}

\newcounter{lecnum}
\renewcommand{\thepage}{\thelecnum-\arabic{page}}
\renewcommand{\thesection}{\thelecnum.\arabic{section}}
\renewcommand{\theequation}{\thelecnum.\arabic{equation}}
\renewcommand{\thefigure}{\thelecnum.\arabic{figure}}
\renewcommand{\thetable}{\thelecnum.\arabic{table}}

\newcommand{\indep}{{\bot\negthickspace\negthickspace\bot}}
\newcommand{\notindep}{{\not\negthickspace\negthinspace{\bot\negthickspace\negthickspace\bot}}}
\newcommand{\definedtobe}{\stackrel{\Delta}{=}}
\renewcommand{\choose}[2]{{{#1}\atopwithdelims(){#2}}}
\newcommand{\argmax}[1]{{\hbox{$\underset{#1}{\mbox{argmax}}\;$}}}
\newcommand{\argmin}[1]{{\hbox{$\underset{#1}{\mbox{argmin}}\;$}}}

%
% The following macro is used to generate the header.
%
\newcommand{\lecture}[4]{
   \pagestyle{myheadings}
   \thispagestyle{plain}
   \newpage
   \setcounter{lecnum}{#1}
   \setcounter{page}{1}
   \noindent
   \begin{center}
   \framebox{
      \vbox{\vspace{2mm}
    \hbox to 6.58in { {\bf EE512~Stochastic Processes
                        \hfill University of Southern California} }
    \hbox to 6.58in { {\bf Fall 2016
                        \hfill Dept. of Electrical Engineering} }
       \vspace{4mm}
       \hbox to 6.28in { {\Large \hfill Lecture #1: #2  \hfill} }
       \vspace{2mm}
       \hbox to 6.28in { {\it Lecturer: {\it Prof: Nayyar {\tt <>}} \hfill Scribe: #3} }
      \vspace{2mm}}
   }
   \end{center}
   \markboth{Lecture #1: #2}{Lecture #1: #2}
   \vspace*{4mm}
}

%
% Convention for citations is authors' initials followed by the year.
% For example, to cite a paper by Leighton and Maggs you would type
% \cite{LM89}, and to cite a paper by Strassen you would type \cite{S69}.
% (To avoid bibliography problems, for now we redefine the \cite command.)
% Also commands that create a suitable format for the reference list.
\renewcommand{\cite}[1]{[#1]}
\def\beginrefs{\begin{list}%
        {[\arabic{equation}]}{\usecounter{equation}
         \setlength{\leftmargin}{2.0truecm}\setlength{\labelsep}{0.4truecm}%
         \setlength{\labelwidth}{1.6truecm}}}
\def\endrefs{\end{list}}
\def\bibentry#1{\item[\hbox{[#1]}]}

%Use this command for a figure; it puts a figure in wherever you want it.
%usage: \fig{NUMBER}{CAPTION}{.eps FILE TO INCLUDE}{WIDTH-IN-INCHES}
\newcommand{\fig}[4]{
			\begin{center}
	                \includegraphics[width=#4,clip=true]{#3} \\
			Figure \thelecnum.#1:~#2
			\end{center}
	}
% Use these for theorems, lemmas, proofs, etc.
\newtheorem{theorem}{Theorem}[lecnum]
\newtheorem{lemma}[theorem]{Lemma}
\newtheorem{proposition}[theorem]{Proposition}
\newtheorem{claim}[theorem]{Claim}
\newtheorem{corollary}[theorem]{Corollary}
\newtheorem{definition}[theorem]{Definition}
% \newenvironment{proof}{{\bf Proof:}}{\hfill\rule{2mm}{2mm}}

% **** IF YOU WANT TO DEFINE ADDITIONAL MACROS FOR YOURSELF, PUT THEM HERE:

\begin{document}
%FILL IN THE RIGHT INFO.
%\lecture{**LECTURE-NUMBER**}{**DATE**}{**LECTURER**}{**SCRIBE**}
\lecture{4}{Sep 1, 2016}{Saket Choudhary}
%\footnotetext{These notes are partially based on those of Nigel Mansell.}

% **** YOUR NOTES GO HERE:

% Some general latex examples and examples making use of the
% macros follow.  
%**** IN GENERAL, BE BRIEF AND COMPLETE. 
% **** THIS ENDS THE EXAMPLES. DON'T DELETE THE FOLLOWING LINE:

Example: \textbf{Branching Processes}
Day 0: $\dots$ (organisms)

Day 1: $Y_1, Y_2, $ (Number of offsprings of $i^{th}$ bacteria)

$\{ Y_i \}$ are iid.

$ Y_i \in \{0,2,4\}$ $p_0+p_2+p_4=1$

Day n $\implies$ $M$ bacteria $(X_n=m)$

$X_n = \#$ of bacteria on day $n$

$P(X_{n+1}=j|X_n=M) = P(Y_1 + Y_2 + \dots +  Y_M = j | X_n =M) = P(X_{n+1} = j | X_n=M, X_{n-1}=a, \dots X_0=2)$

so a branching process is  markov chain

\section*{Multistep Transition Probability}
$P(X_{n+m}=j|X_n=1,X_{n-1}) = P(X_{m}=j|X_{0}=i)$
$m-step$ transition probabilities does on depend on past given the currrent state.

\textbf{Gambler's ruin}: time homegenous $P(X_1=1|X_0=0) = P(X_{101}=1|X_100=0)$


Example: $S=\{1,2,3\}$

$p = \begin{pmatrix}
0.8 & 0.1 & 0.1\\
0.1 & 0.8 & 0.1\\
0.1 & 0.1 & 0.8
\end{pmatrix}$

$P(X_2=1,X_1=3|X_0=2) = P(X_2+1|X_1=3)P(X_1=3|X_0=2)$

$P(A \cap B) = P(A|B)P(B) \implies P(A \cap B|C) = P(A|B \cap C) P(B|C)$

$P(X_2=1,X_1=3|X_0=2) = P(X_2=1|X_1=3)P(X_1=3|X_0=2) = 0.1 * 0.1 = 0.01$


$P(X_2=1|X_0=2) = \sum_i P(X_2=1,X_1=i|X_0=2)P(X_i=i|X_0=2)$


$P(X_2 =j | X_0 =i)$ denoted by $p^2(i,j)$

$p^2(i,j) = \sum_{k \in S} P(X_2=j, X_1=k|X_0=i) = \sum_{k \in S} P(X_2=j|X_1=k) P(X_1=k|X_0=i) = \sum_{k \in S} p(k,j)p(i,k)$

$p^2 = p . p$ Two steo is product of one step.

So $m$ step transition probabilities are denoted by $p^m(i,j)$ 
= $P(X_m=j|X_0=i)$

$m-step$ transition probability matrix is the $m^[th]$ power of one step transition probability matrix.


\subsection{Champan Kplmogorov equations}

$p^{m+n}(i,j) = \sum_{k \in S} p^m(i,k)p^n(k,j)$
 
$P(X_{m+n}=j|X_0=i)$ 

\textbf{Proof}: 

\begin{align*}
P(X_{n+m}=k|X_0=i) &= \sum_{k \in S} P(X_{n+m}=j, X_m=k| X_0=i)\\
&= \sum_{k \in S} P(X_{n+m}=j | X_m=k, X_0=i)P(X_m=k|X_0=i)\\
&= \sum_{k \in S} P(X_{n+m}=j | X_m=k)P(X_m=k|X_0=i)\\
&= \sum_{k \in S} p^n(k,j)p^m(i,k)\\
\end{align*}

\subsection*{Notation}
For some event $A$, $P(A|X_0=x) = P_x(A)$

For some RV Y, $E[Y|X_0=x] = E_x[Y]$

\subsection*{Stopping Time}

\textbf{Definition} Consider a RV T, takes avalues in the set $ \{0,1,2,3 \}$. We say that $T$ is a stopping time with respect to the stochastic process. $\{X_n, n \geq 0 \}$ if the value of $I_{N}$[INCOMPLETE]

$\{X_n, n \geq 0\}$ -- Gambler's ruin problem

$ T = \min\{k \geq 0: X_k \geq 5\}$


Example: $X_0 = 10$ then $T=0$

Calculate $I_{\{T_n\}}$ from $X_0, X_1, \dots X_n$

If $\{T =0 \}$ $\implies$ $ \{X_0 \geq 5 \}$

$\{ T= 1\} \implies \{X_0 <5, X_1 \geq 5 \}$

$ \{ T=n \} \implies \{X_0 <5, X_1<5, \dots X_{n-1} < 5, X_n \geq 5\}$

$T$ is a stopping time

Strong markov property: 

Suppose $T$ is a stopping time with respect to the Markov Chain, $\{X_n, n \geq 0\}$ Given that stopping time $T=n$ and $X_T=y$

Any other information about $X_0, X_1, \dots X_{T-1}$ is irrelevant for predicting future. 

$P(X_{T+1} =j | T=n, X_T=y) = P(X_1=j|X_0=y) = p(y,j)$

$P(X_{T+1} = j|T_n,X_T=y,X_{T-1}=x, \dots X_o=z) = P(X_{T+1}=j | T=n, X_T=y)P(y,j)$


$P(X_{T+m} = j|T=n, X_T=y, X_{T-1}=x) = P(X_{T+m} = j|T=n, X_T=y) = P(X_m=j|X_0=y) = p^m(y,j)$




\end{document}



