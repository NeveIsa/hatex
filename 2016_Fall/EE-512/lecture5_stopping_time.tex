\documentclass{article}
\usepackage{times,amsmath,amsthm,amsfonts,eucal,graphicx, amssymb}
 \setlength{\oddsidemargin}{0.25 in}
\setlength{\evensidemargin}{-0.25 in}
\setlength{\topmargin}{-0.6 in}
\setlength{\textwidth}{6.5 in}
\setlength{\textheight}{8.5 in}
\setlength{\headsep}{0.75 in}
\setlength{\parindent}{0 in}
\setlength{\parskip}{0.1 in}

\newcounter{lecnum}
\renewcommand{\thepage}{\thelecnum-\arabic{page}}
\renewcommand{\thesection}{\thelecnum.\arabic{section}}
\renewcommand{\theequation}{\thelecnum.\arabic{equation}}
\renewcommand{\thefigure}{\thelecnum.\arabic{figure}}
\renewcommand{\thetable}{\thelecnum.\arabic{table}}

\newcommand{\indep}{{\bot\negthickspace\negthickspace\bot}}
\newcommand{\notindep}{{\not\negthickspace\negthinspace{\bot\negthickspace\negthickspace\bot}}}
\newcommand{\definedtobe}{\stackrel{\Delta}{=}}
\renewcommand{\choose}[2]{{{#1}\atopwithdelims(){#2}}}
\newcommand{\argmax}[1]{{\hbox{$\underset{#1}{\mbox{argmax}}\;$}}}
\newcommand{\argmin}[1]{{\hbox{$\underset{#1}{\mbox{argmin}}\;$}}}

%
% The following macro is used to generate the header.
%
\newcommand{\lecture}[4]{
   \pagestyle{myheadings}
   \thispagestyle{plain}
   \newpage
   \setcounter{lecnum}{#1}
   \setcounter{page}{1}
   \noindent
   \begin{center}
   \framebox{
      \vbox{\vspace{2mm}
    \hbox to 6.58in { {\bf EE512~Stochastic Processes
                        \hfill University of Southern California} }
    \hbox to 6.58in { {\bf Fall 2016
                        \hfill Dept. of Electrical Engineering} }
       \vspace{4mm}
       \hbox to 6.28in { {\Large \hfill Lecture #1: #2  \hfill} }
       \vspace{2mm}
       \hbox to 6.28in { {\it Lecturer: {\it Prof: Nayyar {\tt <>}} \hfill Scribe: #3} }
      \vspace{2mm}}
   }
   \end{center}
   \markboth{Lecture #1: #2}{Lecture #1: #2}
   \vspace*{4mm}
}

%
% Convention for citations is authors' initials followed by the year.
% For example, to cite a paper by Leighton and Maggs you would type
% \cite{LM89}, and to cite a paper by Strassen you would type \cite{S69}.
% (To avoid bibliography problems, for now we redefine the \cite command.)
% Also commands that create a suitable format for the reference list.
\renewcommand{\cite}[1]{[#1]}
\def\beginrefs{\begin{list}%
        {[\arabic{equation}]}{\usecounter{equation}
         \setlength{\leftmargin}{2.0truecm}\setlength{\labelsep}{0.4truecm}%
         \setlength{\labelwidth}{1.6truecm}}}
\def\endrefs{\end{list}}
\def\bibentry#1{\item[\hbox{[#1]}]}

%Use this command for a figure; it puts a figure in wherever you want it.
%usage: \fig{NUMBER}{CAPTION}{.eps FILE TO INCLUDE}{WIDTH-IN-INCHES}
\newcommand{\fig}[4]{
			\begin{center}
	                \includegraphics[width=#4,clip=true]{#3} \\
			Figure \thelecnum.#1:~#2
			\end{center}
	}
% Use these for theorems, lemmas, proofs, etc.
\newtheorem{theorem}{Theorem}[lecnum]
\newtheorem{lemma}[theorem]{Lemma}
\newtheorem{proposition}[theorem]{Proposition}
\newtheorem{claim}[theorem]{Claim}
\newtheorem{corollary}[theorem]{Corollary}
\newtheorem{definition}[theorem]{Definition}
% \newenvironment{proof}{{\bf Proof:}}{\hfill\rule{2mm}{2mm}}

% **** IF YOU WANT TO DEFINE ADDITIONAL MACROS FOR YOURSELF, PUT THEM HERE:

\begin{document}
%FILL IN THE RIGHT INFO.
%\lecture{**LECTURE-NUMBER**}{**DATE**}{**LECTURER**}{**SCRIBE**}
\lecture{5}{Sep 06, 2016}{Saket Choudhary}
%\footnotetext{These notes are partially based on those of Nigel Mansell.}

% **** YOUR NOTES GO HERE:

% Some general latex examples and examples making use of the
% macros follow.  
%**** IN GENERAL, BE BRIEF AND COMPLETE. 
% **** THIS ENDS THE EXAMPLES. DON'T DELETE THE FOLLOWING LINE:

NOTE: Durrett's book for Markov Chain.

\textbf{Stopping time}

$T$ is a stopping time wrt $\{ X_n \}_{n \geq 0}$ if $\{ T= n \}$
can be determined from $X_0, X_1, \dots X_n$

\textbf{Strong markov property}


$P(X_{T+m}=z|X_Y=y,T=n) = P(X_m=z|X_0=y) = p^m(z)$


\textbf{Time of first return}

State space of MC is finite.

For  a state $y \in S$ , $T_y = \min \{ n\geq 1: X_n=y \}$

since $n \geq 1$ so $X_0$ is not relevant here.

If $X_n \neq y$ for any finite $n \geq 1$, then we say that $T_y = \infty$

Is $T_y$ a stopping time?

For $I_{T_y=1}$ all we need to know is $X_1$ $[=y, \neq y]$

If $\{T_y =k \}$ then $X_1,X_2, X_{k-1} \neq y$ and $X_k =y$

$P_y(T_y < \infty) = P(T_y < \infty | X_0=y)$ 

Time to return to $y$ is finite given start $X_0=y$ .

$P(T_y < \infty | X_0 = y) = \sum_{n \geq 1} P(T_y=n|X_0=y) = \rho_{yy}$

$\rho_{yy}$: Probability of 1st return to $y$ happening in finite time given $X_0=y$

\textbf{Time of $2^{nd}$ return}

$T_y^2 = \min\{n>T_1: X_n = y|X_0=y  \}$

$P(T_y^2 < \infty) = ?$

$\{T_y^2 M \infty \} \subset \{ T_y < \infty \}$

Thus, $P_y(T_y^2 < \infty) \leq \rho_{yy}$

$P(T_y^2 < \infty) = P(T_y^2 < \infty, T_y^1 < \infty) + P(T_y^2 < \infty, T_y^1=\infty) = P(T_y^2 < \infty , T_y^1 < \infty) = P(Y_y^2 < \infty | T_y^1 < \infty)P(T_y^1 < \infty) = \rho_{yy}^2$


Inductively, $P(T_y^k < \infty) = \rho_{yy}^k$

Two cases:

\begin{itemize}
\item $\rho_{yy}=1$ In this case we say that $y$ is a RS[Recurrent State]. $P_y(T_y^1 < \infty) = 1$ so guaranteed to come back to $y$
$P_y(T_y^k < \infty) = 1$. Number of times the MC visits given $X_0=y$ is infinite
\item 
\end{itemize}

Let $N(y)$ = Total number of visits to $y$ from time 1 onwards
$\{N(y) = \infty \} = \cup_{k \geq 0} \{N(y) \geq k \}$  

$P_y(N(y) = \infty)  = P(\cup_{k \geq 0} \{N(y) \geq k \})
= \lim_{k \longrightarrow \infty} P_y(N(y) > k) = \lim_{k \longrightarrow \infty } P(T_y^k < \infty) = \lim_{k \longrightarrow \infty} \rho_{yy}^k = 1$

IF $\rho_{yy} < 1$ then 
$P(T_y' < \infty) < 1$ and $P(T_y' = \infty ) >0 $


States with $\rho_{yy}<1$ are called transient state

\begin{align*}
P(N(y) = \infty) &= P_y(\cap \{ N(y) \geq k\}\\
&= \lim_{k \longrightarrow \infty} P(T_y^k < infty)\\
&= \rho_{yy}^k\\
&= 0
\end{align*}

For recurrent states: $N_y = \infty$ with probability 1.
for transient states, $N_y < \infty$ with probability 1.

\textbf{Example}

$\rho_{00}=1$ so state $0$ is recurrent[also absorbing]

In general any absorbing state is also recurrent.



%$P(T_1^\infty 
Simimarly state 4 is also recurrant.

If we start at 1, we visit 1 only finitel maaaaaaaaa
Recurring states do not have to be absorbing states.


\textbf{Example 2}
A-A : 0.5
A:B : 0.5
B-B : 1

Both $0,1$ are recurrent

$P(T_0^1 =k) = \frac{1}{2^k}$

$P(T_0^1 < \infty) = 1$ Thus, $0$ is a recurring state, but not an absorbing state.

\textbf{Recurrent and Transient state}

$\rho_{xy}=  P(T_y < \infty | X_0=x) = P_x(T_y<\infty) $

if $\rho_{xy}=0 \implies P_x(T_y = \infty) = 1$

E.g A-A = 1
B-A = 0.6
B-C=0.4
C-C=1

$\rho_{10} = 0.6$

Suppose there exists a finite number $m$ such that the $m$ step 
transition probab from $x$ to $y$ is positivie

$p^m(x,y) > 0 $, then $\rho_{xy} >0$


Converse: Suppose $\rho_{xy} > 0$ then there exists a finite number $
m$ such that $p^m(x,y) > 0$

$\rho_{xy} = P_x(T_y < \infty) = \sum_{m=1}P_x(T_y=m) > 0 \implies $ there exists at least one $m=n$ such that $P_x(T_y=n) > 0$ 

$P_x(T_y=n) \leq P_x(X_n=y)$ $\implies$  $p^n(xy) < 0$

If $\rho_{xy} > 0$ then we say that $x,y$ communicate or $x \longrightarrow y$


\textbf{Theorem} If $\rho_{xy} > 0$ and $\rho_{yx} < 1$ then $x$ is a transient state.

Example: 

$\rho_{21} = 1$

$\rho_{12}  = 0$

So $2$ is transient



\end{document}



