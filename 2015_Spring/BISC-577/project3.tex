%%%%%%%%%%%%%%%%%%%%%%%%%%%%%%%%%%%%%%%%%
% Structured General Purpose Assignment
% LaTeX Template
%
% This template has been downloaded from:
% http://www.latextemplates.com
%
% Original author:
% Ted Pavlic (http://www.tedpavlic.com)
%
% Note:
% The \lipsum[#] commands throughout this template generate dummy text
% to fill the template out. These commands should all be removed when 
% writing assignment content.
%
%%%%%%%%%%%%%%%%%%%%%%%%%%%%%%%%%%%%%%%%%

%----------------------------------------------------------------------------------------
%	PACKAGES AND OTHER DOCUMENT CONFIGURATIONS
%----------------------------------------------------------------------------------------

\documentclass{article}

\usepackage{fancyhdr} % Required for custom headers
\usepackage{lastpage} % Required to determine the last page for the footer
\usepackage{extramarks} % Required for headers and footers
\usepackage{graphicx} % Required to insert images
\usepackage{latexsym}
\usepackage{mathtools}

\usepackage{lipsum} % Used for inserting dummy 'Lorem ipsum' text into the template
%\usepackage[]{algorithm2e}
%\usepackage{algorithmicx}
%\usepackage{algorithm}
%\usepackage{algorithm}
%\usepackage{algorithmic}
%\usepackage{algpseudocode}
%\usepackage{algcompatible}


\usepackage{algorithm}
\usepackage{algorithmic}
%\usepackage{algorithmicx}


%\usepackage{algpseudocode}

%\usepackage{algpseudocode}

%\usepackage[noend]{algpseudocode}
\renewcommand{\algorithmicrequire}{\textbf{Input:}}
\renewcommand{\algorithmicensure}{\textbf{Output:}}
\newcommand{\algorithmicbreak}{\textbf{break}}
\newcommand{\algorithmicgiven}{\textbf{Given:}}
\newcommand{\BREAK}{\STATE \algorithmicbreak}
\newcommand{\GIVEN}{\STATEx \algorithmicgiven}
%\def\NoNumber#1{{\def\alglinenumber##1{}\State #1}\addtocounter{ALG@line}{-1}}

\usepackage{amsmath}
%\usepackage{multline}

% Margins
\topmargin=-0.45in
\evensidemargin=0in
\oddsidemargin=0in
\textwidth=6.5in
\textheight=9.0in
\headsep=0.25in 

\linespread{1.1} % Line spacing

% Set up the header and footer
\pagestyle{fancy}
\lhead{\hmwkAuthorName} % Top left header
\chead{\hmwkClass\ : \hmwkTitle} % Top center header
\rhead{\firstxmark} % Top right header
\lfoot{\lastxmark} % Bottom left footer
\cfoot{} % Bottom center footer
\rfoot{Page\ \thepage\ of\ \pageref{LastPage}} % Bottom right footer
\renewcommand\headrulewidth{0.4pt} % Size of the header rule
\renewcommand\footrulewidth{0.4pt} % Size of the footer rule

\setlength\parindent{0pt} % Removes all indentation from paragraphs

%----------------------------------------------------------------------------------------
%	DOCUMENT STRUCTURE COMMANDS
%	Skip this unless you know what you're doing
%----------------------------------------------------------------------------------------

% Header and footer for when a page split occurs within a problem environment
\newcommand{\enterProblemHeader}[1]{
\nobreak\extramarks{#1}{#1 continued on next page\ldots}\nobreak
\nobreak\extramarks{#1 (continued)}{#1 continued on next page\ldots}\nobreak
}

% Header and footer for when a page split occurs between problem environments
\newcommand{\exitProblemHeader}[1]{
\nobreak\extramarks{#1 (continued)}{#1 continued on next page\ldots}\nobreak
\nobreak\extramarks{#1}{}\nobreak
}

\setcounter{secnumdepth}{0} % Removes default section numbers
\newcounter{homeworkProblemCounter} % Creates a counter to keep track of the number of problems

\newcommand{\homeworkProblemName}{}
\newenvironment{homeworkProblem}[1][Problem \arabic{homeworkProblemCounter}]{ % Makes a new environment called homeworkProblem which takes 1 argument (custom name) but the default is "Problem #"
\stepcounter{homeworkProblemCounter} % Increase counter for number of problems
\renewcommand{\homeworkProblemName}{#1} % Assign \homeworkProblemName the name of the problem
\section{\homeworkProblemName} % Make a section in the document with the custom problem count
\enterProblemHeader{\homeworkProblemName} % Header and footer within the environment
}{
\exitProblemHeader{\homeworkProblemName} % Header and footer after the environment
}

\newcommand{\problemAnswer}[1]{ % Defines the problem answer command with the content as the only argument
\noindent\framebox[\columnwidth][c]{\begin{minipage}{0.98\columnwidth}#1\end{minipage}} % Makes the box around the problem answer and puts the content inside
}

\newcommand{\homeworkSectionName}{}
\newenvironment{homeworkSection}[1]{ % New environment for sections within homework problems, takes 1 argument - the name of the section
\renewcommand{\homeworkSectionName}{#1} % Assign \homeworkSectionName to the name of the section from the environment argument
\subsection{\homeworkSectionName} % Make a subsection with the custom name of the subsection
\enterProblemHeader{\homeworkProblemName\ [\homeworkSectionName]} % Header and footer within the environment
}{
\enterProblemHeader{\homeworkProblemName} % Header and footer after the environment
}
   
%----------------------------------------------------------------------------------------
%	NAME AND CLASS SECTION
%----------------------------------------------------------------------------------------
\DeclarePairedDelimiter\ceil{\lceil}{\rceil}
\DeclarePairedDelimiter\floor{\lfloor}{\rfloor}
\newcommand{\hmwkTitle}{Project\ \# 3 } % Assignment title
\newcommand{\hmwkDueDate}{Tuesday,\ April \ 21,\ 2015} % Due date
\newcommand{\hmwkClass}{BISC-577} % Course/class
\newcommand{\hmwkClassTime}{11:00am} % Class/lecture time
\newcommand{\hmwkAuthorName}{Saket Choudhary} % Your name
\newcommand{\hmwkAuthorID}{2170058637} % Teacher/lecturer
%----------------------------------------------------------------------------------------
%	TITLE PAGE
%----------------------------------------------------------------------------------------

\title{
\vspace{2in}
\textmd{\textbf{\hmwkClass:\ \hmwkTitle}}\\
\normalsize\vspace{0.1in}\small{Due\ on\ \hmwkDueDate}\\
%\vspace{0.1in}\large{\textit{\hmwkClassTime}}
\vspace{3in}
}

\author{\textbf{\hmwkAuthorName} \\
	\textbf{\hmwkAuthorID}
	}
\date{} % Insert date here if you want it to appear below your name

%----------------------------------------------------------------------------------------

\begin{document}

\maketitle

%----------------------------------------------------------------------------------------
%	TABLE OF CONTENTS
%----------------------------------------------------------------------------------------

%\setcounter{tocdepth}{1} % Uncomment this line if you don't want subsections listed in the ToC

\newpage
\tableofcontents
\newpage




\begin{homeworkSection}{Question \# 1} % Section within problem

\problemAnswer{

\textbf{Chip-Seq Experiments:} Chip-Seq experiment couples chromatin immunoprecipitation with high throughput DNA sequencing.
It is used for identifying  binding sites of transcription factors and for idetifying histone related modifications. 'Chip' step
involves cross-linking proteins and DNA making the proteins immobile. This is followed by fragmentation  through sonication/endonucleases, generatiing 100-300bp fragments. The protein of interest is then enriched using a specific antibody that is know to bind selectively to just this
protein. The separated fragments contain(mostly) of the protein bound DNA sequences. DNA can be separated by reversing the cross-links which can then be analyzed for its abundance, computationally. To obtain sufficient signal, $large$(10M+) number of cells are required.
 
Of the two protocols Nano-Chip-Seq and LinDA, the difference exists at the amplification stage. Nano-Chip-Seq
makes use of custom primers during PCR amplification containing a specific restriction site that permits direct addition of illumina sequencing adapters 
These primers form a hairpin structure preventing self-annealing.

LinDA on the other hand makes use of an RNA polymerase from the T7 bacteriophage. 

Since Nano-Chip-Seq relies on PCR amplification and custom primers, these experiments might have technical bias or probably even
overrepresentation of primer sequnces which should be checked for in the data analysis stage.

 
 
}

\end{homeworkSection}

\begin{homeworkSection}{Question \# 2}
	\problemAnswer{ 
	One of the major sources of bias in Chip-Seq studies arises due to the fragmentation step. Fragmentation is necessary to ensure
	only fragments bound to protein are purified. Sonication is known to be more effective in open chromatin regions and hence 
	regions flainkin euchromatin  will shear easily than heterochromatin regions. Transcription factors bind more easily to the open chromatin region which also shears easily and hence gives rise to preferential bias.
	
	"Input" Dna protocol involves isolation of sample that has been crosslinked and sonicated but not immunoprecipitated. 
	The "IgG" control is a "mock" Chip 	reaction that is guranteed to be random. It works by using a 'control' antibody that will bind to non-nuclear proteins randomly.
	
	Presence of controls("input" or "IgG") can be used to estimate 'background' rate of non-specific binding for transcription factors/histones
	which then can be used to filter out the false positive peaks from the analysis samples.
	
	%\textbf{Accession:} \url{http://www.ncbi.nlm.nih.gov/sra/SRX541599[accn]}
	
	The control dataset is an "input" control	drawn from the human esophageal epithelial cell line and thus was islotaed post 
	sonification(without immunoprecipitating)
	
		}
\end{homeworkSection}


\begin{homeworkSection}{Question \# 3}
	\problemAnswer{
	\textbf{H3K9Me3}: http://www.ncbi.nlm.nih.gov/sra/SRX849433[accn] \\
	\textbf{Run}: SRR1768294\\
	\textbf{SRA size}: 349M\\
	\textbf{FastQ size}: 2.6G\\
	Single End reads of 36bp size.\\

	\textbf{Total Reads}:  18400047\\
	
	\textbf{H3K4Me3}: http://www.ncbi.nlm.nih.gov/sra/SRX849427[accn] \\
	\textbf{Run}: SRR1768267	\\
	\textbf{SRA size}: 439M\\
		\textbf{FastQ size}: 3.4G\\
	Single End reads of 36bp size.\\
\textbf{	Total Reads}: 23514026\\
	
	\textbf{Project Name}: Conserved epigenomic signatures between mouse and human elucidate immune basis of Alzheimer's disease (house mouse)
	http://www.ncbi.nlm.nih.gov/bioproject/PRJNA273302
	
	\textbf{Organism}: Mouse

		}
\end{homeworkSection}


\begin{homeworkSection}{Question \# 4}
	\problemAnswer{
		\textbf{SRR1768267.fastq} \\
	Total Reads:  23514026\\
Unaligned Reads: 1426753 (6.07\%)% aligned 0 times \\
Aligend Reads(Exactly once): 17620103 (74.93\%)\\ aligned exactly 1 time\\
Aligend Reads(Aligned more than once): 4467170 (19.00\%) aligned >1 times\\
Total alignment rate: 93.93\% overall alignment rate\\
Time: 18m1.391s \\
Sam file size: 4G \\


 \textbf{SRR1768294.fastq} \\
  Total reads: 18400047\\
 Unaligned Reads: 1440302 (7.83\%)\\% aligned 0 times
 Aligend Reads(Exactly once): 9027568 (49.06\%) aligned exactly 1 time \\
 Aligend Reads(Aligned more than once): 7932177 (43.11\%) aligned >1 times \\
 Total alignment rate: 92.17\% overall alignment rate \\
Time    21m3.115s\\
Sam file size: 3.1G \\


Mapped to mm10.
		
		}
\end{homeworkSection}

\begin{homeworkSection}{Question \# 5}
	\problemAnswer{
	Program used: MACS(v1.4)\\
	
	Parameters comnfigurable:\\
	
	1. $gsize$: Genome size of the organism. This is made use in the $p-value$ calculations and hence may impact
	the number of peaks depending on the threshold. It is more likely to see peaks in a smaller genome than a large one. \\
	
	2. $pvalue$: $p-value$ cut-off for defining a peak. Default is $10^{-5}$, but more stringent cut-offs might be required for noisier datasets\\
	
	3. $nolambda$: MACS models reads distribution as poisson distribution. A "control" if present can be used to estimate the 'background' $\lambda$. If no control is present, the background $\lambda$ is fixed.\\
	
	4. $nomodel$: MACS models the shifting size of Chip-Seq tags(which often are shifted to $3'$ end, this size being unknown) to precisely locate the binding sites, which might be difficult to model in case of Chip-Seq's broad peaks. So though a precise location is possible by enabling $nomodel$, the data might not necessarily show bimodal pattern.\\


Each file tool around one minute to run. The output were bedGraph and bed files.

Bedgraph files are tab delimited files that stores in each row the chromosome number, start position, end position and the read count mapping to these positions. These positions are a superset of the positions appearing in the bed files.

Bed files are also tab delimited with the first column as the chromosome positions the next two as the start and end positions of peak in that chromosome
and the fourth column as the $-10 \log(10pvalue)$. The summits file has the height information for peaks.
	
	
	
	
	
		}
	
\end{homeworkSection}
 
 
  \begin{homeworkSection}{Question \# 6}
  	\problemAnswer{
  		\textbf{H3K9Me3}\\
  		Number of peaks: 3338\\
  		Mean peak length: 1185.777\\
  		Median peak length: 814.5\\
  		Max: 46335 (Could be all noise)\\
  		
  		\textbf{H3K4Me3}\\
		NMumber of peaks: 28823\\
		Mean peak length: 1981.701\\
		Median peak length: 1594\\
		Max: 54633 (Could be all noise)\\
		
		The number of peaks detected in H3K9Me3 state are less than that in
		H3K4Me3, thus pointing that H3K9Me3 impacts the heterochromatin(compact) region while H3K4Me3 must be associated with 
		euchromatin	thus indicating H3K4Me should be associated with actication while H3K9Me3 might be asociated with repression.
		The datasets do not seem to be high quality because of the inherent noise and 'low peaks' present.
		Mthfs was one of the genes showing a peak for H3K4Me3, pointing that the gene is probably upregulated.
		
  	}
\end{homeworkSection}

 
 	
% \end{homeworkSection}
 
 

 	
 %\end{homeworkSection}

\end{document}
